% Options for packages loaded elsewhere
\PassOptionsToPackage{unicode}{hyperref}
\PassOptionsToPackage{hyphens}{url}
%
\documentclass[
]{article}
\usepackage{lmodern}
\usepackage{amssymb,amsmath}
\usepackage{ifxetex,ifluatex}
\ifnum 0\ifxetex 1\fi\ifluatex 1\fi=0 % if pdftex
  \usepackage[T1]{fontenc}
  \usepackage[utf8]{inputenc}
  \usepackage{textcomp} % provide euro and other symbols
\else % if luatex or xetex
  \usepackage{unicode-math}
  \defaultfontfeatures{Scale=MatchLowercase}
  \defaultfontfeatures[\rmfamily]{Ligatures=TeX,Scale=1}
\fi
% Use upquote if available, for straight quotes in verbatim environments
\IfFileExists{upquote.sty}{\usepackage{upquote}}{}
\IfFileExists{microtype.sty}{% use microtype if available
  \usepackage[]{microtype}
  \UseMicrotypeSet[protrusion]{basicmath} % disable protrusion for tt fonts
}{}
\makeatletter
\@ifundefined{KOMAClassName}{% if non-KOMA class
  \IfFileExists{parskip.sty}{%
    \usepackage{parskip}
  }{% else
    \setlength{\parindent}{0pt}
    \setlength{\parskip}{6pt plus 2pt minus 1pt}}
}{% if KOMA class
  \KOMAoptions{parskip=half}}
\makeatother
\usepackage{xcolor}
\IfFileExists{xurl.sty}{\usepackage{xurl}}{} % add URL line breaks if available
\IfFileExists{bookmark.sty}{\usepackage{bookmark}}{\usepackage{hyperref}}
\hypersetup{
  pdftitle={Gender triggers the low voter turnout?},
  pdfauthor={Linzi Guan},
  hidelinks,
  pdfcreator={LaTeX via pandoc}}
\urlstyle{same} % disable monospaced font for URLs
\usepackage[margin=1in]{geometry}
\usepackage{graphicx,grffile}
\makeatletter
\def\maxwidth{\ifdim\Gin@nat@width>\linewidth\linewidth\else\Gin@nat@width\fi}
\def\maxheight{\ifdim\Gin@nat@height>\textheight\textheight\else\Gin@nat@height\fi}
\makeatother
% Scale images if necessary, so that they will not overflow the page
% margins by default, and it is still possible to overwrite the defaults
% using explicit options in \includegraphics[width, height, ...]{}
\setkeys{Gin}{width=\maxwidth,height=\maxheight,keepaspectratio}
% Set default figure placement to htbp
\makeatletter
\def\fps@figure{htbp}
\makeatother
\setlength{\emergencystretch}{3em} % prevent overfull lines
\providecommand{\tightlist}{%
  \setlength{\itemsep}{0pt}\setlength{\parskip}{0pt}}
\setcounter{secnumdepth}{-\maxdimen} % remove section numbering

\title{Gender triggers the low voter turnout?\thanks{Code and data are available at:
\url{https://github.com/linziguan0118/304-Final-Project.git}}}
\usepackage{etoolbox}
\makeatletter
\providecommand{\subtitle}[1]{% add subtitle to \maketitle
  \apptocmd{\@title}{\par {\large #1 \par}}{}{}
}
\makeatother
\subtitle{Men are more likely to vote!}
\author{Linzi Guan}
\date{23 December 2020}

\begin{document}
\maketitle
\begin{abstract}
Observing the significant political influence of low voter turnout and
huge different between the predicted results of election in polls and in
real world, this paper analyses the issues of low voter turnout and aims
to find people with what features are more unwilling to vote. Datasets
about the 2019 Canadian Election survey are obtained, cleaned and
analysed with a simple and multiple logistic regression model, with the
interest of gender impact, I found that men are more likely to vote than
women and others. The results and the study appeal for politicans to
predict the final results of the election taking everyones' vote and
thus to come up policies that call people to vote.

\textbf{Keywords:} Political participation, Low voter turnout, 2019 US
eletion, logistic regression
\end{abstract}

\hypertarget{introduction}{%
\section{Introduction}\label{introduction}}

Political participation has always been a sharp issue in the elections
and low voter turnout and inequalities have significant political and
policy consequences. Due to the low voter turnout, the results of public
opinion poll have large gap and sometimes even huge difference from the
true, final results. Eyes on the issues of low voter turnout and the
difference betwwen poll and the actual results, a question to be asked
is people with what features are more unwilling to vote. In this paper,
a simple logistic regression model is conducted according to the CES
dataset obtained from the 2019 Canadian election in R and findings on
the importance of turnout based on the model and results are discussed.

Based on the dataset, I have applied a statistical method to build one
simple logistic regression model in this analysis to predict find people
with what features are more unwilling to vote in 2019 Canadian election.
I have learned that men are more likely to vote than women and other
genders after conducting simple and multiple linear regressions.

The paper is organized in the following parts with a full disclosure and
analysis of the data I used to built my study in the Data Section, some
detailed discussions on the statistical models that I used for
forecasting in Model Section, some discussions and results, and also
some limitations and nextsteps.

\hypertarget{data}{%
\section{Data}\label{data}}

CES2019 online data is used in this paper. The data are obtained in
Canada Election Study 2019, which is conducted online.

\hypertarget{data-variables}{%
\subsubsection{Data Variables}\label{data-variables}}

Variables I am using include: cps19\_gender: A Categorical variable
indicating self-reported gender, including a man, a woman and Other
(e.g.~Trans, non-binary, two-spirit, gender-queer) cps19\_v\_likely: A
Categorical variable indicating whether or not people are willing to
vote cps19\_education: A Categorical variable indicating education level
cps19\_bornin\_canada: A Categorical variable indicating whether or not
the person is born in Canada

\hypertarget{survey-methodology}{%
\subsubsection{Survey methodology}\label{survey-methodology}}

The 2019 Canadian Election Study was conducted with a two-wave panel
with a modified rolling-cross section during the campaign period and a
post-election recontact wave.

\hypertarget{population-frame-and-sample}{%
\subsubsection{Population, frame and
sample}\label{population-frame-and-sample}}

The target population of Canadian citizens and permanent residents who
are aged 18 or older. Frame is people who can access Qualtrics and the
sample is designed to be online sample with 37,822 members of the
Canadian general population through Qualtrics, which targets stratified
by region and balanced on gender and age within each region.

\hypertarget{data-features-and-strengths}{%
\subsubsection{Data features and
strengths}\label{data-features-and-strengths}}

\begin{itemize}
\tightlist
\item
  Representativeness: The proportion of respondents are controlled and
  targeted to be representative: for example, 50\% men and 50\% women
  are targeted.
\item
  Data accuracy: Duplicate variables have been removed to improve data
  accuracy.
\item
  Non response: Non response answers and answers that ineligible or
  incomplete are removed to increase the overall accuracy.
\end{itemize}

\hypertarget{data-weaknesses}{%
\subsubsection{Data weaknesses}\label{data-weaknesses}}

\begin{itemize}
\tightlist
\item
  Imperfect coverage: The sample only considers people that can access
  Qualtric, which cannot cover all the people that have the voting
  rights. +Sampling error: Sampling error is unavoidable. The results
  will vary from sample to sample. It will surely be different from the
  results of the true voting data.
\end{itemize}

\hypertarget{key-facts-about-the-data}{%
\subsubsection{Key facts about the
data:}\label{key-facts-about-the-data}}

✓ Overall, most people indicate that they are certain to vote in the
election but there are also a number of people indicating not to vote or
unlikely to vote. ✓ There are more women taking the survey than men. But
there is no obvious proclivity for voting. ✓ Different education level
has its own potential to voting choice. ✓ Proportion of not voting is
higher for people born in Canada

*Figure 1: Voting intentions. The length of the bars represent the
magnitude of number of voting people. Longer bar means larger number of
votes. From the bar chart, we can find most people indicate that they
are certain to vote in the election but there are also a number of
people indicating not to vote or unlikely to vote.

*Figure 2: Voting intention counts in different gender. The colours
represent different genders: red for man, green for women and blue for
others. The length of the bars represent the magnitude of number of
voting people. Longer bar means larger number of votes. From the bar
chart, we can find here are more women taking the survey than men. But
there is no obvious proclivity for voting.

*Figure 3: Voting intention counts in different education level. The
colours represent different education levels. The length of the bars
represent the magnitude of number of voting people. Longer bar means
larger number of votes. Different education level has its own potential
to voting choice.

*Figure 4: Voting intention counts with different immigration status.
The colours represent different immigration statuses. The length of the
bars represent the magnitude of number of voting people. Longer bar
means larger number of votes. Proportion of not voting is higher for
people born in Canada

\hypertarget{model}{%
\section{Model}\label{model}}

To see how gender impacts the likelihood to vote, I firstly conduct a
simple linear regression model of certain to vote on gender. To improve
the model, I conduct a multiple linear regression model of certain to
vote on gender, after controlling education and immigration status.

\hypertarget{result}{%
\section{Result}\label{result}}

The result of simple linear regression of certain to vote on gender:

The results of multiple regression of certain to vote on gender, holding
education and immigration status constant are shown below:

\hypertarget{discussion}{%
\section{Discussion}\label{discussion}}

According to the simple model result, the adjusted R square is less than
1\%, showing that less than 1\% of variation in certain to vote can be
explained by gender. P-value for a woman and others are both smaller
than 5\%, and that is to say they are statistically significant. On
average, men are 0.02 more likely to vote than women and 0.06 more
likely to vote than others.

With multiple regression, the adjusted R square has been improved to
2\%, showing more variation could be explained by gender, holding other
variables constant. P-values are both smaller than 5\%, and that is to
say they are statistically significant. On average, holding other
variables constant, men are 0.02 more likely to vote than women and 0.05
more likely to vote than others.

\hypertarget{limitations-and-next-steps}{%
\section{Limitations and next steps}\label{limitations-and-next-steps}}

The coverage of the survey is limited so the result is not fully
representative. The model does not fit that well, so next step I might
use multilevel logistic regression with post stratification.

\hypertarget{reference}{%
\section{Reference}\label{reference}}

Hadley Wickham, Jim Hester and Winston Chang (2020). devtools: Tools to
Make Developing R Packages Easier. \url{https://devtools.r-lib.org/},
\url{https://github.com/r-lib/devtools}.

Jacob Kaplan (2020). fastDummies: Fast Creation of Dummy (Binary)
Columns and Rows from Categorical Variables.
\url{https://github.com/jacobkap/fastDummies},
\url{https://jacobkap.github.io/fastDummies/}.

Joseph Larmarange (2020). labelled: Manipulating Labelled Data. R
package version 2.7.0. \url{http://larmarange.github.io/labelled/}

Paul A. Hodgetts and Rohan Alexander (2020). cesR: Access the CES
Datasets a Little Easier.. R package version 0.1.0.

R Core Team (2020). R: A language and environment for statistical
computing. R Foundation for Statistical Computing, Vienna, Austria. URL
\url{https://www.R-project.org/}.

Stephenson, Laura B., Allison Harell, Daniel Rubenson and Peter John
Loewen. The 2019 Canadian Election Study -- Online Collection.
{[}dataset{]}

Wickham et al., (2019). Welcome to the tidyverse. Journal of Open Source
Software, 4(43), 1686, \url{https://doi.org/10.21105/joss.01686}

\end{document}
